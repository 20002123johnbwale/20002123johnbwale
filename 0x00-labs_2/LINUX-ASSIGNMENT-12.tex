\documentclass{article}

\begin{document}

\section*{Shell Scripting Questions}

\begin{enumerate}
  \item The purpose of the \texttt{-x} shell option, enabled with \texttt{set -x}, is to:
  \begin{enumerate}[label=(\Alph*)]
    \item print each command as given
    \item stop the script when an error occurs
    \item print each command after expansion
    \item prevent overwriting of files by redirection
  \end{enumerate}
  \textbf{Answer: (C) print each command after expansion}

  \item The conditional \texttt{if [ "\$a" == b ]; then} tests whether:
  \begin{enumerate}[label=(\Alph*)]
    \item Two processes are running the same program.
    \item Two files have the same modification date.
    \item Two files have the same contents.
    \item Two strings are equal.
  \end{enumerate}
  \textbf{Answer: (D) Two strings are equal}

  \item The conditional \texttt{if [ -a "\$file" ]; then} tests whether:
  \begin{enumerate}[label=(\Alph*)]
    \item A file is a hidden file.
    \item A file has been accessed.
    \item A file is readable by all users.
    \item A file exists.
  \end{enumerate}
  \textbf{Answer: (D) A file exists}

  \item The conditional \texttt{if [ "\$file1" -nt "\$file2" ]; then} tests whether:
  \begin{enumerate}[label=(\Alph*)]
    \item One file is neater than another.
    \item One file depends on another.
    \item One file is newer than another.
    \item One file is larger than another.
  \end{enumerate}
  \textbf{Answer: (C) One file is newer than another}

  \item The purpose of the \texttt{-v} shell option, enabled with \texttt{set -v}, is to:
  \begin{enumerate}[label=(\Alph*)]
    \item print each command after expansion
    \item stop the script when an error occurs
    \item print each command as given
    \item prevent overwriting of files by redirection
  \end{enumerate}
  \textbf{Answer: (C) print each command as given}

  \item The purpose of the \texttt{-e} shell option, enabled with \texttt{set -e}, is to:
  \begin{enumerate}[label=(\Alph*)]
    \item print each command after expansion
    \item print each command as given
    \item prevent overwriting of files by redirection
    \item stop the script when an error occurs
  \end{enumerate}
  \textbf{Answer: (D) stop the script when an error occurs}

  \item In a directory containing the files \texttt{a.txt}, \texttt{b.txt}, \texttt{c.java}, and \texttt{d.cpp}, how many iterations will the loop \texttt{for f in *.txt; do} execute?
  \begin{enumerate}[label=(\Alph*)]
    \item 4
    \item 2
    \item 3
    \item 1
  \end{enumerate}
  \textbf{Answer: (B) 2 iterations}

  \item In a directory containing the files \texttt{a.txt}, \texttt{b.txt}, \texttt{c.java}, and \texttt{d.cpp}, how many iterations will the loop \texttt{for f in \$(find -name *java); do} execute?
  \begin{enumerate}[label=(\Alph*)]
    \item 1
    \item 2
    \item 4
    \item 3
  \end{enumerate}
  \textbf{Answer: (C) 4 iterations}

  \item The purpose of the \texttt{set} command is to:
  \begin{enumerate}[label=(\Alph*)]
    \item enable shell options
    \item assign values to shell variables
    \item assign values to environment variables
    \item set permissions on files
  \end{enumerate}
  \textbf{Answer: (A) enable shell options}

  \item The conditional \texttt{if command; then} tests whether:
  \begin{enumerate}[label=(\Alph*)]
    \item The command runs successfully, with a zero return code.
    \item The command is the name of the currently-running script.
    \item The command is running in the background.
    \item The command exists.
  \end{enumerate}
  \textbf{Answer: (A) The command runs successfully, with a zero return code.}

  \item The conditional \texttt{if [ -z "\$a" ]; then} tests whether:
  \begin{enumerate}[label=(\Alph*)]
    \item A jobs list is empty.
    \item A file is empty.
    \item A string is empty.
    \item A process list is empty.
  \end{enumerate}
  \textbf{Answer: (C) A string is empty.}
\end{enumerate}

\end{document}
