\documentclass{article}

\begin{document}

\title{Makefile Questions}
\author{JOHN BWALE}
\date{\today}

\maketitle

\section*{Makefile Questions}

\begin{enumerate}
    \item A makefile describes the relationship between a collection of files.
    \begin{enumerate}
        \item goals
        \item order
        \item dependencies
        \item rules
    \end{enumerate}
    \textbf{Answer:} C. dependencies

    \item When running make, the command line arguments should be:
    \begin{enumerate}
        \item Files that have changed since the last time make was run.
        \item A list of commands that make should execute.
        \item Files that should be updated (i.e. what you want).
        \item Actually, make does not accept command line arguments.
    \end{enumerate}
    \textbf{Answer:} A. Files that have changed since the last time make was run.

    \item The purpose of the make command is to:
    \begin{enumerate}
        \item search for and act on files in or below a directory.
        \item execute commands to create or update files, based on the rules in a makefile.
        \item create a new file in the current directory.
        \item find lines that match a pattern.
    \end{enumerate}
    \textbf{Answer:} B. execute commands to create or update files, based on the rules in a makefile.

    \item Recall this explanation for the meaning of a makefile rule: "When [blank] is changed, run the command to update [blank]." The first blank in this explanation should be:
    \begin{enumerate}
        \item prerequisite
        \item recipe
        \item target
        \item None of the above
    \end{enumerate}
    \textbf{Answer:} C. target

    \item Here is an example makefile rule.
    \begin{verbatim}
    aaa.txt: bbb.txt
        ccc bbb.txt > aaa.txt
    \end{verbatim}
    In this example, the filename aaa.txt on the first line is the rule's:
    \begin{enumerate}
        \item recipe
        \item prerequisite
        \item target
        \item None of the above
    \end{enumerate}
    \textbf{Answer:} C. target

    \item Recall this explanation for the meaning of a makefile rule: "When [blank] is changed, run the command to update [blank]." The second blank in this explanation should be:
    \begin{enumerate}
        \item target
        \item prerequisite
        \item recipe
        \item None of the above
    \end{enumerate}
    \textbf{Answer:} B. prerequisite

    \item Recall this explanation for the meaning of a makefile rule: "When [blank] is changed, run the command to update [blank]." The third blank in this explanation should be:
    \begin{enumerate}
        \item recipe
        \item prerequisite
        \item target
        \item None of the above
    \end{enumerate}
    \textbf{Answer:} A. recipe

    \item Here is an example makefile rule.
    \begin{verbatim}
    aaa.txt: bbb.txt
        ccc bbb.txt > aaa.txt
    \end{verbatim}
    In this example, when does make execute the command on the second line?
    \begin{enumerate}
        \item when bbb.txt is older than aaa.txt
        \item when bbb.txt is newer than aaa.txt
        \item when ccc cannot be found
        \item when bbb.txt does not exist
    \end{enumerate}
    \textbf{Answer:} B. when bbb.txt is newer than aaa.txt

    \item The purpose of the touch command is to:
    \begin{enumerate}
        \item Change a file's owner to be the current user.
        \item Show a file's access history.
        \item Change a file's modification time to the current time.
        \item Show a file's permissions.
    \end{enumerate}
    \textbf{Answer:} C. Change a file's modification time to the current time.

    \item Here is an example makefile rule.
    \begin{verbatim}
    aaa.txt: bbb.txt
        ccc bbb.txt > aaa.txt
    \end{verbatim}
    In this example, the command \texttt{ccc bbb.txt > aaa.txt} is the rule's:
    \begin{enumerate}
        \item recipe
        \item target
        \item prerequisite
        \item None of the above
    \end{enumerate}
    \textbf{Answer:} A. recipe

    \item Here is an example makefile rule.
    \begin{verbatim}
    aaa.txt: bbb.txt
        ccc bbb.txt > aaa.txt
    \end{verbatim}
    In this example, the filename bbb.txt on the
\end{enumerate}

\end{document}
