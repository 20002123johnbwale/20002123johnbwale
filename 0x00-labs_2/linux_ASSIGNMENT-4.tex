\documentclass{article}

\title{LINUX}
\author{JOHN BWALE}


\begin{document}
\maketitle

\textbf{1. Which option can be used with the \texttt{ls} command to show details including permissions, owner, group, size in bytes, and modification date about each file?}

\begin{itemize}
  \item[A.] \texttt{-h}
  \item[B.] \texttt{-a}
  \item[C.] \texttt{-l}
  \item[D.] \texttt{-v}
\end{itemize}

\textbf{Answer:} C. \texttt{-l}

\vspace{0.5cm}

\textbf{2. The purpose of the \texttt{less} command is to}

\begin{itemize}
  \item[A.] reduce the number of devices connected to the computer
  \item[B.] show a list of files in a directory, including hidden and deleted files
  \item[C.] show a list of running processes in the system, including processes owned by other users
  \item[D.] show the contents of a file interactively, allowing scrolling and searching
\end{itemize}

\textbf{Answer:} D. show the contents of a file interactively, allowing scrolling and searching

\vspace{0.5cm}

\textbf{3. Which of these refers to the home directory?}

\begin{itemize}
  \item[A.] \texttt{.}
  \item[B.] \texttt{/}
  \item[C.] \texttt{..}
  \item[D.] \texttt{\~}
\end{itemize}

\textbf{Answer:} D. \texttt{\~}

\vspace{0.5cm}

\textbf{4. The purpose of the \texttt{man} command is to}

\begin{itemize}
  \item[A.] show the path to the executable of a command
  \item[B.] show the history for a command
  \item[C.] show the documentation for a command
  \item[D.] show the version for a command
\end{itemize}

\textbf{Answer:} C. show the documentation for a command

\vspace{0.5cm}

\textbf{5. The command to show a list of files and subdirectories is}

\begin{itemize}
  \item[A.] \texttt{show-files}
  \item[B.] \texttt{1s}, as in 'one-ess'
  \item[C.] \texttt{ls}, as in 'ell-ess'
  \item[D.] \texttt{file-show}
\end{itemize}

\textbf{Answer:} C. \texttt{ls}, as in 'ell-ess'

\vspace{0.5cm}

\textbf{6. The purpose of the \texttt{pwd} command is to}

\begin{itemize}
  \item[A.] Show the contents of the root directory.
  \item[B.] Show the name of the home directory.
  \item[C.] Show the contents of the current directory.
  \item[D.] Show the name of the current directory.
\end{itemize}

\textbf{Answer:} D. Show the name of the current directory.

\vspace{0.5cm}

\textbf{7. Which command will change the current directory to the user's home directory?}

\begin{itemize}
  \item[A.] \texttt{pwd \~}
  \item[B.] \texttt{cd \~}
  \item[C.] \texttt{pwd ..}
  \item[D.] \texttt{cd ..}
\end{itemize}

\textbf{Answer:} B. \texttt{cd \~}

\vspace{0.5cm}

\textbf{8. The command \texttt{ls \~{}/..} would show a list of all}

\begin{itemize}
  \item[A.] home directories on the system
  \item[B.] files in the current directory
  \item[C.] files in all directories on the system
  \item[D.] files in the user's home directory
\end{itemize}

\textbf{Answer:} A. home directories on the system

\vspace{0.5cm}

\textbf{9. The purpose of the \texttt{cd} command is to}

\begin{itemize}
  \item[A.] Change the current directory.
  \item[B.] Print the name of the current directory.
  \item[C.] Change the home directory.
  \item[D.] Clear the disk.
\end{itemize}

\textbf{Answer:} A. Change the current directory.

\vspace{0.5cm}

\textbf{10. The purpose of the \texttt{cat} command is to}

\begin{itemize}
  \item[A.] display the contents of a file
  \item[B.] edit a file
  \item[C.] categorize a file
  \item[D.] move a file
\end{itemize}

\textbf{Answer:} A. display the contents of a file

\vspace{0.5cm}

\textbf{11. Within \texttt{less}, which command will search forward through the file for the word 'public'?}

\begin{itemize}
  \item[A.] :public
  \item[B.] \textbackslash public
  \item[C.] /public
  \item[D.] public
\end{itemize}

\textbf{Answer:} C. /public

% Add more answers below this line

\end{document}