\documentclass{article}
\begin{document}
\title{Sample Final Exam}
\author{JOHN BWALE}
\date{\today}
\maketitle

\section{Multiple Choice Questions}

\begin{enumerate}
    \item Which of these commands will use command substitution to copy the executable for the program grep to the current directory?
    \begin{enumerate}
        \item \verb|cp ?(grep) .|
        \item \verb|cp (which grep) .|
        \item \verb|cp $(grep) .|
        \item \verb|cp $(which grep) .|
    \end{enumerate}
    \textbf{Answer:} D. \verb|cp $(which grep) .|

    \item In a directory containing many kinds of files, which of these commands will display at most 10 files with the extension cpp?
    \begin{enumerate}
        \item \verb|ls ?.cpp | head|
        \item \verb|ls .cpp | head|
        \item \verb|ls *.cpp | head|
        \item \verb|ls *.cpp | tac|
    \end{enumerate}
    \textbf{Answer:} C. \verb|ls *.cpp | head|

    \item Which of these files would be listed by the command \verb|ls a b c d e f|?
    \begin{enumerate}
        \item Six files named a, b, c, d, e, and f.
        \item A single file named 'a b c d e f g', with 5 spaces in its name.
        \item Two files, one named 'a b c' and one named 'd e f', each with two spaces in its name.
        \item None of the above.
    \end{enumerate}
    \textbf{Answer:} D. None of the above.

    \item Which symbol is used to force the next character to be treated normally?
    \begin{enumerate}
        \item \verb|`|
        \item \verb|?|
        \item \verb|\|
        \item \verb|/|
    \end{enumerate}
    \textbf{Answer:} C. \verb|\|

    \item For the command \verb|ls hi.?| which files would be listed?
    \begin{enumerate}
        \item \verb|hi.java|
        \item \verb|hi.py|
        \item \verb|hi.|
        \item \verb|hi.c|
    \end{enumerate}
    \textbf{Answer:} A. \verb|hi.java| and B. \verb|hi.py|

    \item Which of these commands will list all of the files whose name contains an opening parenthesis?
    \begin{enumerate}
        \item \verb|ls *(*|
        \item \verb|ls *\(\*|
        \item \verb|ls *\(*|
        \item \verb|ls (|
    \end{enumerate}
    \textbf{Answer:} B. \verb|ls *\(\*|

    \item Which of these files would be listed by the command \verb|ls hello.*|?
    \begin{enumerate}
        \item \verb|hello.py|
        \item \verb|hello.csharp|
        \item \verb|hello.cpp|
        \item All of the above.
    \end{enumerate}
    \textbf{Answer:} D. All of the above.

    \item What would the effect of the command \verb|ls *.c| be?
    \begin{enumerate}
        \item To list all files whose names consist of exactly one character, followed by a period, followed by a \verb|c|.
        \item To list all files with a \verb|c| extension.
        \item To list all files whose names do not contain a \verb|c|.
        \item To list all files whose names contain a \verb|c|.
    \end{enumerate}
    \textbf{Answer:} B. To list all files with a \verb|c| extension.

    \item Which of these files would be listed by the command \verb|ls a\ b\ c\ d\ e\ f|, which has a space after each of its backslashes?
    \begin{enumerate}
        \item Six files named a, b, c, d, e, and f.
        \item A single file named 'a b c d e f', with 5 spaces in its name.
        \item Two files, one named 'a b c' and one named 'd e f', each with two spaces in its name.
        \item None of the above.
    \end{enumerate}
    \textbf{Answer:} A. Six files named a, b, c, d, e, and f.

    \item Which of the following is not a valid way to write a comment in Python?
    \begin{enumerate}
        \item \verb|# This is a comment|
        \item \verb|// This is a comment|
        \item \verb|'''This is a comment'''|
        \item \verb|"This is a comment"|
    \end{enumerate}
    \textbf{Answer:} B. \verb|// This is a comment|

    \item What will be the output of the following Python code?

    \begin{verbatim}
    def my_func(x):
        if x < 0:
            return "Negative"
        elif x == 0:
            return "Zero"
        else:
            return "Positive"

    print(my_func(10))
    \end{verbatim}

    \begin{enumerate}
        \item Negative
        \item Zero
        \item Positive
        \item Error: Invalid function call
    \end{enumerate}
    \textbf{Answer:} C. Positive

    \item Which of the following data types in Python is mutable?
    \begin{enumerate}
        \item int
        \item str
        \item list
        \item tuple
    \end{enumerate}
    \textbf{Answer:} C. list

    \item What is the output of the following Python code?

    \begin{verbatim}
    my_list = [1, 2, 3, 4, 5]
    print(my_list[1:3])
    \end{verbatim}

    \begin{enumerate}
        \item [1, 2]
        \item [2, 3]
        \item [2, 3, 4]
        \item [1, 3]
    \end{enumerate}
    \textbf{Answer:} B. [2, 3]

    \item Which of the following is the correct syntax to open a file named "data.txt" in Python for reading?
    \begin{enumerate}
        \item \verb|file = open("data.txt", "r")|
        \item \verb|file = open("data.txt", "w")|
        \item \verb|file = open("data.txt", "a")|
        \item \verb|file = open("data.txt", "x")|
    \end{enumerate}
    \textbf{Answer:} A. \verb|file = open("data.txt", "r")|

    \item What is the output of the following Python code?

    \begin{verbatim}
    my_dict = {"apple": 1, "banana": 2, "cherry": 3}
    del my_dict["banana"]
    print(len(my_dict))
    \end{verbatim}

    \begin{enumerate}
        \item 0
        \item 1
        \item 2
        \item 3
    \end{enumerate}
    \textbf{Answer:} C. 2

    \item Which of the following is the correct way to define a class named "Person" in Python with a constructor that takes two parameters: "name" and "age"?
    \begin{enumerate}
        \item 
        \begin{verbatim}
        class Person:
            def __init__(self, name, age):
                self.name = name
                self.age = age
        \end{verbatim}
        \item 
        \begin{verbatim}
        class Person(name, age):
            def __init__(self, name, age):
                self.name = name
                self.age = age
        \end{verbatim}
        \item 
        \begin{verbatim}
        class Person:
            def __init__(name, age):
                self.name = name
                self.age = age
        \end{verbatim}
        \item 
        \begin{verbatim}
        class Person:
            def __init__(self):
                self.name = name
                self.age = age
        \end{verbatim}
    \end{enumerate}
    \textbf{Answer:} A. 
    \begin{verbatim}
    class Person:
        def __init__(self, name, age):
            self.name = name
            self.age = age
    \end{verbatim}
\end{enumerate}

\end{document}